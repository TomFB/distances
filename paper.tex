\documentclass{amsart}
\usepackage{pifont,latexsym,ifthen,amsthm,calc,amsfonts,amssymb,amsbsy,amsmath}
\usepackage[square,comma]{natbib}
\newcommand{\abs}[1]{\left\lvert #1\right\rvert}
\newcommand{\brac}[1]{\left( #1\right)}
\newcommand{\norm}[1]{\left\lVert #1\right\rVert}
\newcommand{\ffield}{\mathbb{F}_q(\!(t^{-1})\!)}
\newtheorem{theorem}{Theorem}[section]
\newtheorem{lemma}[theorem]{Lemma}
\newtheorem{corollary}[theorem]{Corollary}

\begin{document}

\title{Distinct distances in function fields}

\author{Thomas F. Bloom and Timothy G. F. Jones}

\begin{abstract}
We will prove a lower bound for the number of distinct distances in function fields.
\end{abstract}  

\maketitle
\tableofcontents

Everything should happen in this document for clarity. If you have a hairbrained idea, want to start a new discussion, or anything like, add a new section. 

Let's not bother with the PDF updates - I assume we can both compile the \LaTeX file ourselves without much effort.

\section{Aims}
We hope to prove a lower bound for the number of distinct distances in $\ffield$ by adapting the proof of Szekeley.

\section{Definitions and notation}
We just work with the one-dimensional case for now. Fix a prime power $q=p^r$ and the completion of the function field $\ffield$. We consider a finite set $P\subset\ffield$. The non-archimedean norm is defined by $\abs{x}=q^{\deg(x)}$. The distance between two points is $\abs{x-y}$. The set of distinct distances between elements of $P$ is
\[D(P)=\{ \abs{x-y} : (x,y)\in P^2\}.\]
For brevity let $d(P)=\abs{D(P)}$. 

A balls is a set of the form
  \[B(x,r)=\left\{y \in \ffield : \norm{x-y}\leq r\right\}.\]
We will call $r\in\mathbb{R}$ the radius of the ball $B(x,r)$. The following lemma will be useful.

\begin{lemma}\label{theorem:nonarchball}
If $B_1$ and $B_2$ are balls in $\ffield$ then either they are disjoint, or $B_1 \subset B_2$, or $B_2 \subset B_1$. If in addition $B_1$ and $B_2$ have the same radius then either they are disjoint or $B_1=B_2$.
\end{lemma}

A finite set $A \subset \ffield$ is {\em separable} if its elements can be indexed as
	\[A=\left\{a_1,\ldots,a_{|A|}\right\}\]
in such a way that for each $1 \leq j \leq |A|$ there is a ball $B_j$ with 
	\[A \cap B_j =\left\{a_1,\ldots,a_j\right\}.\] 
	
\section{Proof sketch}
A sketch will go here and be built upon. Tim should copy his latex file into this section.

\subsection{Lower bounds}

\subsection{Upper bounds}

\section{Applications}
A linear code of length $n$ and rank $k$ is a linear subspace $C\leq \mathbb{F}_q^n$ of dimension $k$. The distance $d$ of a linear code is the minimum distance between distinct codewords. This is the Hamming distance, i.e. the number of elements in which they differ (focus on $q=2$ for now). 

An $[n,k,d]$ code is a linear code of length $n$, dimension $k$ and distance $d$.
\end{document}
